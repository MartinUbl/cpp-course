% !TEX TS-program = pdflatex
% !TEX encoding = UTF-8 Unicode

\documentclass{beamer}
\usepackage[czech]{babel}
\usepackage[utf8]{inputenc}
\usepackage{times}
\usepackage[T1]{fontenc}
\usepackage{verbatim}
\usepackage{listings}
\usepackage{xcolor}

\mode<presentation>
{
	\usetheme{antibes}
	\usecolortheme{orchid}
}

\setbeamertemplate{navigation symbols}{}

\definecolor{cmtgreen}{RGB}{0,192,0}

\begin{document}

\title{C++; delete Java;}
\subtitle{Část 4: významné STL struktury}
\author{Kennny}
\date{srpen 2017}

\frame{\titlepage}

\lstset{language=C++,
        basicstyle=\ttfamily,
        keywordstyle=\color{blue}\ttfamily,
        stringstyle=\color{red}\ttfamily,
        commentstyle=\color{cmtgreen}\ttfamily,
        morecomment=[l][\color{magenta}]{\#}
}

\newenvironment{xframe}[1][]
  {\begin{frame}[fragile,environment=xframe,#1]}
  {\end{frame}}

\begin{comment}
\begin{xframe}{tttt}
	\begin{itemize}
		\item
	\end{itemize}
\end{xframe}
\end{comment}



\section{STL}
\subsection{Obecné}



\begin{xframe}{Významné struktury}
	\begin{itemize}
		\item \texttt{string} - klasický řetězec
		\item \texttt{vector} - dynamické pole
		\item \texttt{list} - spojový seznam
		\item \texttt{map} - key-value úložiště (red-black strom, seřazeno)
		\item \texttt{set} - množina (seřazená)
		\item \texttt{unordered\_map} - key-value úložiště (hash tabulka, neřazeno)
		\item \texttt{unordered\_set} - množina (neřazená)
	\end{itemize}
\end{xframe}

\begin{xframe}{Významné struktury}
	\begin{itemize}
		\item \texttt{queue} - fronta
		\item \texttt{stack} - zásobník
		\item \texttt{deque} - obousměrná fronta
		\item \texttt{priority\_queue} - prioritní fronta
		\item a další...
	\end{itemize}
\end{xframe}

\subsection{Konkrétní struktury}

\begin{xframe}{std::string}
	\begin{itemize}
		\item \texttt{\#include <string>}
		\item obaluje dynamickou práci s řetězci
		\item odstiňuje nutnou realokaci, přetěžuje operátor +
		\item implementuje pomocné metody (substr, find, ...)
		\item pro porovnání přetěžuje operátory == a !=
	\end{itemize}
\end{xframe}

\begin{xframe}{Příklad}
	\begin{itemize}
		\item Prostor pro příklad 04\_a\_string
	\end{itemize}
\end{xframe}

\begin{xframe}{Vsuvka: iterátor}
	\begin{itemize}
		\item jednotný obalový prvek sloužící pro průchod
		\item každý STL kontejner ho má
		\item zpravidla se získává metodou \texttt{begin()} (popř. \texttt{rbegin()}, pokud to kontejner podporuje)
		\item přetěžuje operátor ++ (posun na další prvek v kontejneru)
		\item některé přetěžují operátor + (posun o N prvků v kontejneru), pokud to daný typ podporuje
		\item prvek na pozici iterátoru se získá dereferencí (přetěžuje unární operátor*)
\begin{lstlisting}[basicstyle=\fontsize{8}{9}\selectfont\ttfamily]
int hodnota = *iter;
\end{lstlisting}
	\end{itemize}
\end{xframe}

\begin{xframe}{std::vector}
	\begin{itemize}
		\item \texttt{\#include <vector>}
		\item šablonový typ
		\item obaluje dynamické pole daného typu
\begin{lstlisting}[basicstyle=\fontsize{8}{9}\selectfont\ttfamily]
std::vector<float> fVector;
std::vector<int> iVector;
std::vector<std::string> stringVector;
\end{lstlisting}
		\item rezervuje prostor "napřed"
		\item přetěžuje operátor []
		\item vnitřně jde o souvislou paměť
	\end{itemize}
\end{xframe}

\begin{xframe}{std::vector}
	\begin{itemize}
		\item vkládání na konec (O(1))
			\begin{itemize}
				\item metodou \texttt{push\_back}
				\item metodou \texttt{resize} a přes index
				\item metodou \texttt{reserve} a následně \texttt{push\_back}
			\end{itemize}
		\item vkládání obecné (O(n))
			\begin{itemize}
				\item metodou \texttt{insert}
			\end{itemize}
		\item mazání z konce (O(1))
			\begin{itemize}
				\item metodou \texttt{pop\_back}
			\end{itemize}
		\item mazání obecné (O(n))
			\begin{itemize}
				\item metodou \texttt{erase} pomocí iterátoru
			\end{itemize}
		\item přístup k prvkům (O(1))
			\begin{itemize}
				\item metodou \texttt{at()}
				\item operátorem []
			\end{itemize}
		\item průchod (O(n))
			\begin{itemize}
				\item přes indexy od 0 až do \texttt{size()}
				\item iterátory od \texttt{begin()} až do \texttt{end()}
				\item range-based for
			\end{itemize}
	\end{itemize}
\end{xframe}

\begin{xframe}{Příklad}
	\begin{itemize}
		\item Prostor pro příklad 04\_b\_vector
	\end{itemize}
\end{xframe}


\begin{xframe}{std::list}
	\begin{itemize}
		\item \texttt{\#include <list>}
		\item šablonový typ
		\item obaluje spojový seznam daného typu
\begin{lstlisting}[basicstyle=\fontsize{8}{9}\selectfont\ttfamily]
std::list<float> fList;
std::list<int> iList;
std::list<std::string> stringList;
\end{lstlisting}
		\item analogie k LinkedList z Javy
	\end{itemize}
\end{xframe}

\begin{xframe}{std::list}
	\begin{itemize}
		\item vkládání (O(1))
			\begin{itemize}
				\item metodou \texttt{push\_back} (na konec)
				\item metodou \texttt{insert} (kamkoliv)
			\end{itemize}
		\item mazání (O(1))
			\begin{itemize}
				\item metodou \texttt{erase} pomocí iterátoru nebo rozsahu iterátorů
			\end{itemize}
		\item přístup k prvkům (O(n))
			\begin{itemize}
				\item iterace od začátku ruční
				\item \texttt{begin()} a \texttt{std::advance}
			\end{itemize}
		\item průchod (O(n))
			\begin{itemize}
				\item iterátory od \texttt{begin()} až do \texttt{end()}
				\item range-based for
			\end{itemize}
	\end{itemize}
\end{xframe}

\begin{xframe}{Příklad}
	\begin{itemize}
		\item Prostor pro příklad 04\_c\_list
	\end{itemize}
\end{xframe}


\begin{xframe}{std::map}
	\begin{itemize}
		\item \texttt{\#include <map>}
		\item šablonový typ
		\item obaluje key-value úložiště zadaných typů
\begin{lstlisting}[basicstyle=\fontsize{8}{9}\selectfont\ttfamily]
std::map<int, float> ifMap;
std::map<long, std::string> intStringMap;
std::map<std::string, int> stringIntMap;
\end{lstlisting}
		\item implementace red-black stromem
		\item pro uložení musí typ být porovnatelný operátorem <
			\begin{itemize}
				\item např. int, long, std::string toto splňují (std::string přetěžuje operátor <)
				\item pozor ale na \texttt{const char*} - jde o adresu, tedy číslo
			\end{itemize}
	\end{itemize}
\end{xframe}

\begin{xframe}{std::map}
	\begin{itemize}
		\item vkládání (O(log n))
			\begin{itemize}
				\item užitím klíče a operátoru []
				\item metodou \texttt{insert} (vkládat pár)
			\end{itemize}
		\item mazání (O(log n))
			\begin{itemize}
				\item metodou \texttt{erase} (klíč nebo iterátor)
			\end{itemize}
		\item přístup k prvkům (O(log n))
			\begin{itemize}
				\item užitím klíče a operátoru [] (! pokud neexistuje, vytvoří ho !)
				\item získání iterátoru metodou \texttt{find()}
			\end{itemize}
		\item průchod (O(n))
			\begin{itemize}
				\item iterátory od \texttt{begin()} až do \texttt{end()}
				\item range-based for
			\end{itemize}
	\end{itemize}
\end{xframe}

\begin{xframe}{std::map}
	\begin{itemize}
		\item co když klíč nemá operátor < přímo, nemá ho ani přetížen, nebo jednoduše vyžadujeme jiné chování?
		\item --> dodefinujeme si buď operátor, nebo definujeme jiný komparátor mapy
		\item operátor< - známe
		\item komparátor - funkce, lambda funkce nebo funktor (struktura s přetíženým operátorem())
\begin{lstlisting}[basicstyle=\fontsize{8}{9}\selectfont\ttfamily]
// const char* komparacni funktor pro mapu
struct constchar_comparator
{
    bool operator()(const char *a, const char *b) const
    {
        return strcmp(a, b) < 0;
    }
};
\end{lstlisting}
		\item prvky jsou stejné, pokud \texttt{!(a < b || b < a)}
	\end{itemize}
\end{xframe}

\begin{xframe}{Příklad}
	\begin{itemize}
		\item Prostor pro příklad 04\_d\_map
	\end{itemize}
\end{xframe}



\begin{xframe}{std::set}
	\begin{itemize}
		\item \texttt{\#include <set>}
		\item šablonový typ
		\item obaluje množinu daného typu
\begin{lstlisting}[basicstyle=\fontsize{8}{9}\selectfont\ttfamily]
std::set<float> fSet; // muze byt nesikovne (float aritmetika)!
std::set<int> iSet;
std::set<std::string> stringSet;
\end{lstlisting}
	\end{itemize}
\end{xframe}

\begin{xframe}{std::set}
	\begin{itemize}
		\item vkládání (O(log n))
			\begin{itemize}
				\item metodou \texttt{insert}
			\end{itemize}
		\item mazání (O(log n))
			\begin{itemize}
				\item metodou \texttt{erase} pomocí klíče nebo iterátoru
			\end{itemize}
		\item detekce přítomnosti prvku (O(log n))
			\begin{itemize}
				\item užitím \texttt{find()}
			\end{itemize}
		\item průchod (O(n))
			\begin{itemize}
				\item iterátory od \texttt{begin()} až do \texttt{end()}
				\item range-based for
			\end{itemize}
	\end{itemize}
\end{xframe}

\begin{xframe}{std::set}
	\begin{itemize}
		\item pro komparátor platí stejná pravidla jako u mapy
		\item není vhodné pro nahrazování vectoru / listu, každý container má svůj účel
	\end{itemize}
\end{xframe}

\begin{xframe}{Příklad}
	\begin{itemize}
		\item Prostor pro příklad 04\_e\_set
	\end{itemize}
\end{xframe}


\begin{xframe}{std::unordered\_map}
	\begin{itemize}
		\item \texttt{\#include <unordered\_map>}
		\item rozhraní stejné jako std::map
		\item rozdíl v implementaci - hash tabulka
		\item v podstatě analogie k HashMap v Javě
		\item přidání, přístup k prvku a mazání průměrně O(1)
		\item nezaručuje pořadí prvků
		\item proti obyčejné std::map zabírá více paměti
	\end{itemize}
\end{xframe}

\begin{xframe}{std::unordered\_set}
	\begin{itemize}
		\item \texttt{\#include <unordered\_set>}
		\item úplně stejná analogie jako std::map a std::unordered\_map
	\end{itemize}
\end{xframe}

\subsection{Vedlejší struktury}

\begin{xframe}{std::queue}
	\begin{itemize}
		\item \texttt{\#include <queue>}
		\item klasická implementace fronty
		\item důležité metody:
			\begin{itemize}
				\item \texttt{push()} - vložit prvek
				\item \texttt{front()} - přední prvek (kandidát na výběr)
				\item \texttt{pop()} - zahození předního prvku
				\item \texttt{empty()} - je fronta prázdná?
				\item \texttt{size()} - počet prvků ve frontě
			\end{itemize}
	\end{itemize}			
\begin{lstlisting}[basicstyle=\fontsize{8}{9}\selectfont\ttfamily]
// vyprazdneni (zpracovani) fronty
while (!q.empty())
{
    TypPrvku& el = q.front();

    // prace s prvkem
    
    ique.pop();
}
\end{lstlisting}
\end{xframe}


\begin{xframe}{std::stack}
	\begin{itemize}
		\item \texttt{\#include <stack>}
		\item klasická implementace zásobníku
		\item důležité metody:
			\begin{itemize}
				\item \texttt{push()} - vložit prvek
				\item \texttt{top()} - prvek na vrcholu (kandidát na výběr)
				\item \texttt{pop()} - zahození předního prvku
				\item \texttt{empty()} - je fronta prázdná?
				\item \texttt{size()} - počet prvků ve frontě
			\end{itemize}
	\end{itemize}			
\begin{lstlisting}[basicstyle=\fontsize{8}{9}\selectfont\ttfamily]
// vyprazdneni (zpracovani) zasobniku
while (!q.empty())
{
    TypPrvku& el = q.top();

    // prace s prvkem
    
    ique.pop();
}
\end{lstlisting}
\end{xframe}

\begin{xframe}{Příklad}
	\begin{itemize}
		\item Prostor pro příklad 04\_f\_misc
	\end{itemize}
\end{xframe}








\begin{xframe}{Konec 4. části}
\texttt{exit(0);}
\end{xframe}




\end{document}




