% !TEX TS-program = pdflatex
% !TEX encoding = UTF-8 Unicode

\documentclass{beamer}
\usepackage[czech]{babel}
\usepackage[utf8]{inputenc}
\usepackage{times}
\usepackage[T1]{fontenc}
\usepackage{verbatim}
\usepackage{listings}
\usepackage{xcolor}

\mode<presentation>
{
	\usetheme{antibes}
	\usecolortheme{orchid}
}

\setbeamertemplate{navigation symbols}{}

\definecolor{cmtgreen}{RGB}{0,192,0}

\begin{document}

\title{C++; delete Java;}
\subtitle{Část 7: vlákna a synchronizace}
\author{Kennny}
\date{srpen 2017}

\frame{\titlepage}

\lstset{language=C++,
        basicstyle=\ttfamily,
        keywordstyle=\color{blue}\ttfamily,
        stringstyle=\color{red}\ttfamily,
        commentstyle=\color{cmtgreen}\ttfamily,
        morecomment=[l][\color{magenta}]{\#}
}

\newenvironment{xframe}[1][]
  {\begin{frame}[fragile,environment=xframe,#1]}
  {\end{frame}}

\begin{comment}
\begin{xframe}{tttt}
	\begin{itemize}
		\item
	\end{itemize}
\end{xframe}
\end{comment}



\section{Vlákna a synchronizace}
\subsection{Vlákno}



\begin{xframe}{Vlákno}
	\begin{itemize}
		\item \texttt{\#include <thread>}
		\item konstruktor vytváří i spouští vlákno
		\item destruktor - pokud má objekt stále přiřazeno fyzické vlákno, hází výjimku
			\begin{itemize}
				\item lze vyřešit např. užitím metody \texttt{detach()}
			\end{itemize}
		\item lze předat vláknu parametry takřka 1:1 s konstruktorem \texttt{std::thread}
		\item konstruktor má jako první argument funkci, kterou má provádět; zbytek argumentů jsou argumenty této funkce pro dané vlákno
		\item metoda \texttt{join()} - "připojí"~aktuální vlákno k tomuto
	\end{itemize}
\end{xframe}

\begin{xframe}{Vlákno}
	\begin{itemize}
		\item Vytvoření vlákna a počkání na ukončení
\begin{lstlisting}[basicstyle=\fontsize{8}{9}\selectfont\ttfamily]
// vytvori vlakno
std::thread vlakno(&zpracujFrontu, 5);

// pocka na jeho ukonceni
vlakno.join();

// napr. zde je uz bezpecne volat destruktor
\end{lstlisting}
		\item Vytvoření vlákna a "odtržení"~od objektu, aby mohl být bezpečně volán destruktor
\begin{lstlisting}[basicstyle=\fontsize{8}{9}\selectfont\ttfamily]
{
    // vytvori vlakno
    std::thread vlakno(&zpracujFrontu, 5);

    // odtrzeni od objektu
    vlakno.detach();
}
\end{lstlisting}
	\end{itemize}
\end{xframe}

\begin{xframe}{Explicitní předání řízení a spánek}
	\begin{itemize}
		\item \texttt{std::this\_thread::yield()} - vzdá se svého časového kvanta ve prospěch jiného vlákna (kterého, to řídí plánovač)
		\item \texttt{std::this\_thread::sleep\_for()} - spánek na určitou dobu
		\item \texttt{std::this\_thread::sleep\_until()} - spánek do určitého momentu
	\end{itemize}
\end{xframe}

\begin{xframe}{Vsuvka: std::chrono}
	\begin{itemize}
		\item \texttt{\#include <chrono>}
		\item modul STL pro práci s časovými úseky a body
		\item datové typy zaobaleny metodami podle jednotek
			\begin{itemize}
				\item např. \texttt{std::chrono::milliseconds()}, \texttt{std::chrono::seconds()}
				\item výsledný typ vždy \texttt{std::chrono::duration}
			\end{itemize}
		\item od C++14 podporuje speciální literály pro eliminaci dlouhé syntaxe
\begin{lstlisting}[basicstyle=\fontsize{8}{9}\selectfont\ttfamily]
using namespace std::chrono_literals;

std::this_thread::sleep_for(1250ms);
\end{lstlisting}
		\item spousty dalších věcí, ale pro nás důležité zatím jen tohle
	\end{itemize}
\end{xframe}

\begin{xframe}{Vsuvka: thread\_local}
	\begin{itemize}
		\item k dispozici je malé úložiště pro vlákno (zásobník)
		\item lze deklarovat proměnnou jako \texttt{thread\_local}
		\item pak má každé vlákno svou vlastní instanci pod stejným jménem
\begin{lstlisting}[basicstyle=\fontsize{8}{9}\selectfont\ttfamily]
thread_local int threadCounter = 0;
\end{lstlisting}
		\item platnost sdružená s kontextem vlákna
	\end{itemize}
\end{xframe}

\begin{xframe}{Příklad}
	\begin{itemize}
		\item Prostor pro příklad 07\_a\_basics
	\end{itemize}
\end{xframe}



\subsection{Synchronizace}

\begin{xframe}{std::mutex}
	\begin{itemize}
		\item \texttt{\#include <mutex>}
		\item klasický mutex pro ochranu kritické sekce
		\item varianty
			\begin{itemize}
				\item \texttt{std::mutex} - klasický mutex
				\item \texttt{std::recursive\_mutex} - mutex který podporuje rekurzivní zamykání ze stejného vlákna
				\item \texttt{std::timed\_mutex} - mutex s podporou timeoutu při čekání
				\item \texttt{std::shared\_timed\_mutex} - sdílený mutex s podporou timeoutu (např. čtenář-písař))
				\item (\texttt{std::shared\_mutex} - sdílený mutex); (C++17)
				\item a další...
			\end{itemize}
	\end{itemize}
\end{xframe}

\begin{xframe}{std::mutex}
	\begin{itemize}
		\item zamykání by mělo být obaleno RAII zámkem
		\item dostupné zámky
			\begin{itemize}
				\item \texttt{std::lock\_guard} - jednoduchý lock/unlock zámek
				\item \texttt{std::unique\_lock} - exkluzivní zámek (lze odložit zamčení, pracovat se sdíleným mutexem, ..)
				\item \texttt{std::shared\_lock} - sdílený zámek (význam u shared mutexů)
			\end{itemize}
		\item pro potřeby kurzu se omezíme na klasický mutex, \texttt{std::lock\_guard} a \texttt{std::unique\_lock}
	\end{itemize}
\end{xframe}

\begin{xframe}{Zamčení mutexu}
	\begin{itemize}
		\item klasické jednoduché použití pro kritickou sekci chráněnou jedním mutexem
\begin{lstlisting}[basicstyle=\fontsize{8}{9}\selectfont\ttfamily]
// sdileny mutex, deklarovan nekde viditelne
std::mutex mtx;

// scope kriticke sekce
{
    std::lock_guard<std::mutex> lck(mtx);
    
    // kriticka sekce
}
// uvolneni zamku s koncem platnosti (scope)
\end{lstlisting}
	\end{itemize}
\end{xframe}

\begin{xframe}{Zamčení více mutexů}
	\begin{itemize}
		\item použití pro (vícenásobnou) kritickou sekci chráněnou více mutexy
\begin{lstlisting}[basicstyle=\fontsize{8}{9}\selectfont\ttfamily]
// sdilene mutexy
std::mutex mtx1, mtx2;

// scope kriticke sekce
{
    std::unique_lock<std::mutex> lck1(mtx, std::defer_lock);
    std::unique_lock<std::mutex> lck2(mtx, std::defer_lock);
    
    // zatim nezamceno
    
    std::lock(lck1, lck2);
    
    // kriticka sekce
}
// uvolneni obou zamku s koncem platnosti (scope)
\end{lstlisting}
	\end{itemize}
\end{xframe}

\begin{xframe}{Příklad}
	\begin{itemize}
		\item Prostor pro příklad 07\_b\_mutex
	\end{itemize}
\end{xframe}


\begin{xframe}{std::condition\_variable}
	\begin{itemize}
		\item \texttt{\#include <condition\_variable>}
		\item klasická podmínková proměnná
		\item k fungování potřebuje zámek a odpovídající mutex
		\item metody
			\begin{itemize}
				\item \texttt{wait()} - čeká dokud není probuzena
				\item \texttt{wait\_for()} - čeká dokud není probuzena nebo neuplyne zadaná doba
				\item \texttt{wait\_until()} - čeká dokud není probuzena nebo do zadaného momentu
				\item \texttt{notify\_one()} - probudí jedno blokované vlákno
				\item \texttt{notify\_all()} - probudí všechny
			\end{itemize}
	\end{itemize}
\end{xframe}

\begin{xframe}{std::condition\_variable}
	\begin{itemize}
		\item \texttt{wait()} jako argument bere zámek a volitelně i podmínku k probuzení ve formě funkce, lambda funkce nebo funktoru
		\item tato podmínka dokáže odstínit spurious wakeup
	\end{itemize}
\begin{lstlisting}[basicstyle=\fontsize{8}{9}\selectfont\ttfamily]
std::unique_lock<std::mutex> lck(mtx);

std::condition_variable cv;

cv.wait(lck, &canWakeUp);

...

bool canWakeUp()
{
    return (count == 0);
}
\end{lstlisting}
\end{xframe}

\begin{xframe}{Příklad}
	\begin{itemize}
		\item Prostor pro příklad 07\_c\_cond\_variables
	\end{itemize}
\end{xframe}

\begin{xframe}{std::atomic}
	\begin{itemize}
		\item \texttt{\#include <atomic>}
		\item šablonový typ nad primitivními datovými typy
		\item zaručuje atomické provedení takto proveditelných operací
			\begin{itemize}
				\item ++, --
				\item +=, -=, \&=, |=, \texttt{\^}=
			\end{itemize}
		\item není potřeba explicitně nic zamykat
	\end{itemize}
\end{xframe}

\begin{xframe}{std::atomic}
	\begin{itemize}
		\item všechny zmíněné operátory jsou přetíženy
		\item není potřeba speciálního zacházení
	\end{itemize}
\begin{lstlisting}[basicstyle=\fontsize{8}{9}\selectfont\ttfamily]
std::atomic<int> myCounter = 0;

myCounter++;
\end{lstlisting}
\end{xframe}

\begin{xframe}{Příklad}
	\begin{itemize}
		\item Prostor pro příklad 07\_d\_atomic
	\end{itemize}
\end{xframe}




\begin{xframe}{Konec 7. části}
\texttt{exit(0);}
\end{xframe}




\end{document}




