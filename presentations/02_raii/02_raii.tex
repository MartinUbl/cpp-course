% !TEX TS-program = pdflatex
% !TEX encoding = UTF-8 Unicode

\documentclass{beamer}
\usepackage[czech]{babel}
\usepackage[utf8]{inputenc}
\usepackage{times}
\usepackage[T1]{fontenc}
\usepackage{verbatim}
\usepackage{listings}
\usepackage{xcolor}

\mode<presentation>
{
	\usetheme{antibes}
	\usecolortheme{orchid}
}

\setbeamertemplate{navigation symbols}{}

\definecolor{cmtgreen}{RGB}{0,192,0}

\begin{document}

\title{C++; delete Java;}
\subtitle{Část 2: RAII}
\author{Kennny}
\date{srpen 2017}

\frame{\titlepage}

\lstset{language=C++,
        basicstyle=\ttfamily,
        keywordstyle=\color{blue}\ttfamily,
        stringstyle=\color{red}\ttfamily,
        commentstyle=\color{cmtgreen}\ttfamily,
        morecomment=[l][\color{magenta}]{\#}
}

\newenvironment{xframe}[1][]
  {\begin{frame}[fragile,environment=xframe,#1]}
  {\end{frame}}

\begin{comment}
\begin{xframe}{tttt}
	\begin{itemize}
		\item
	\end{itemize}
\end{xframe}
\end{comment}



\section{RAII}
\subsection{Význam konstruktoru a destruktoru}



\begin{xframe}{RAII}
	\begin{itemize}
		\item Resource Acquisition Is Initialization
		\item využívá konstruktor a destruktor pro zabrání / uvolnění zdroje
		\item prakticky vždy staticky alokováno
		\item to umožňuje vázat RAII objekt na konkrétní scope
	\end{itemize}
\end{xframe}

\begin{xframe}{RAII}
	\begin{itemize}
		\item je to ochrana proti zapomínání
		\item zpohodlnění práce
		\item často je navíc využita i explicitní scope pro přehlednost
		\item na jednoduchých příkladech není moc vidět důležitost a užitečnost principu RAII
	\end{itemize}
\end{xframe}



\subsection{Příklady}

\begin{xframe}{Příklad kostry RAII struktury}

\begin{lstlisting}[basicstyle=\fontsize{8}{9}\selectfont\ttfamily]
struct WCKeyHolder
{
    public:
        WCKeyHolder()
        {
            acquireKey();
        }

        ~WCKeyHolder()
        {
            releaseKey();
        }
}
\end{lstlisting}

\end{xframe}

\begin{xframe}{Příklad použití RAII struktury}

\begin{lstlisting}[basicstyle=\fontsize{8}{9}\selectfont\ttfamily]
// sraci scope
{
    // konstruktor zabere zdroj
    WCKeyHolder wckey;
    
    sit();
    while (!ass.empty())
        ass.pop();
    wipe(ass);
}
// konec explicitni scope zapricini volani destruktoru
// tedy uvolneni zdroje
\end{lstlisting}

\end{xframe}

\begin{xframe}{RAII}
	\begin{itemize}
		\item spousty věcí v C++ STL nějakým způsobem utilizuje RAII
		\item souborové streamy - otevírá a zavírá soubor
		\item mutex a jeho lock\_guard - zabírá a uvolňuje mutex
		\item smart pointery - zvyšují a snižují reference count
		\item a další...
	\end{itemize}
\end{xframe}

\begin{xframe}{Příklad}
	\begin{itemize}
		\item Prostor pro příklad 02\_a\_raii
	\end{itemize}
\end{xframe}




\begin{xframe}{Konec 2. části}
\texttt{exit(0);}
\end{xframe}




\end{document}




