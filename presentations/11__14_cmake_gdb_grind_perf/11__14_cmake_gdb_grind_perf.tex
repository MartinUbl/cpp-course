% !TEX TS-program = pdflatex
% !TEX encoding = UTF-8 Unicode

\documentclass{beamer}
\usepackage[czech]{babel}
\usepackage[utf8]{inputenc}
\usepackage{times}
\usepackage[T1]{fontenc}
\usepackage{verbatim}
\usepackage{listings}
\usepackage{xcolor}

\mode<presentation>
{
	\usetheme{antibes}
	\usecolortheme{wolverine}
}

\setbeamertemplate{navigation symbols}{}

\begin{document}

\title{C++ support}
\subtitle{Část 11 - 14: CMake, GDB, Valgrind, Perf}
\author{Kennny}
\date{srpen 2017}

\frame{\titlepage}

\lstset{language=bash,
        basicstyle=\ttfamily,
        keywordstyle=\color{blue}\ttfamily,
        stringstyle=\color{red}\ttfamily,
        commentstyle=\color{cmtgreen}\ttfamily,
        morecomment=[l][\color{magenta}]{\#}
}

\newenvironment{xframe}[1][]
  {\begin{frame}[fragile,environment=xframe,#1]}
  {\end{frame}}

\begin{comment}
\begin{xframe}{tttt}
	\begin{itemize}
		\item
	\end{itemize}
\end{xframe}
\end{comment}



\section{CMake}
\subsection{Obecné}

\begin{xframe}{CMake úvod}
	\begin{itemize}
		\item nástroj pro dynamické generování konfigurací k sestavení
		\item generuje MSVS solutions, makefile, Codeblocks projekty, atd...
		\item vždy CLI utilita
		\item nyní i GUI
	\end{itemize}
\end{xframe}

\begin{xframe}{CMakeLists.txt}
	\begin{itemize}
		\item root soubor
		\item zpravidla obsahuje definici min. verze
\begin{lstlisting}[basicstyle=\fontsize{8}{9}\selectfont\ttfamily]
CMAKE_MINIMUM_REQUIRED(VERSION 2.4)
\end{lstlisting}
		\item vždy obsahuje název celého projektu (workspace, solution)
\begin{lstlisting}[basicstyle=\fontsize{8}{9}\selectfont\ttfamily]
PROJECT(my-fancy-cpp-app)
\end{lstlisting}
		\item pro přehlednost je možné, aby podsložky obsahovaly také CMakeLists.txt, který bude načten odděleně; vyvoláme ho přidáním podsložky
\begin{lstlisting}[basicstyle=\fontsize{8}{9}\selectfont\ttfamily]
ADD_SUBDIRECTORY(src)
\end{lstlisting}
		\item nic z toho zatím nemá vliv na strukturu projektu
	\end{itemize}
\end{xframe}

\subsection{Příkazy}

\begin{xframe}{Spustitelný soubor}
	\begin{itemize}
		\item přidání spustitelného souboru k sestavení (resp. projektu pro MSVS v rámci solution)
\begin{lstlisting}[basicstyle=\fontsize{8}{9}\selectfont\ttfamily]
ADD_EXECUTABLE(<target-name> <file list...>)
\end{lstlisting}
		\item např.
\begin{lstlisting}[basicstyle=\fontsize{8}{9}\selectfont\ttfamily]
ADD_EXECUTABLE(my-executable main.cpp module.cpp othermodule.cpp)
\end{lstlisting}
		\item toto vybuildí soubor \texttt{my-executable} (resp. \texttt{my-executable.exe} na Windows)
		\item má další možné parametry, pro základní pochopení nejsou nutné
	\end{itemize}
\end{xframe}

\begin{xframe}{Knihovny}
	\begin{itemize}
		\item přidání knihovny k sestavení (resp. projektu pro MSVS v rámci solution) probíhá analogicky
\begin{lstlisting}[basicstyle=\fontsize{8}{9}\selectfont\ttfamily]
ADD_LIBRARY(<target-name> [library type] <file list...>)
\end{lstlisting}
		\item kde \texttt{[library type]} může být např.:
			\begin{itemize}
				\item \texttt{SHARED} - dynamicky linkovaná knihovna (.dll nebo .so)
				\item \texttt{STATIC} - staticky linkovaná knihovna (.lib nebo .a)
			\end{itemize}
		\item např.
\begin{lstlisting}[basicstyle=\fontsize{8}{9}\selectfont\ttfamily]
ADD_LIBRARY(my-toolset STATIC libmain.cpp a.cpp b.cpp)
\end{lstlisting}
		\item toto vybuildí soubor \texttt{my-toolset.a} (resp. \texttt{my-executable.lib} na Windows)
	\end{itemize}
\end{xframe}

\begin{xframe}{Include cesty}
	\begin{itemize}
		\item include cesty se dají přidat globálně / pro jeden cíl
\begin{lstlisting}[basicstyle=\fontsize{8}{9}\selectfont\ttfamily]
INCLUDE_DIRECTORIES(<dirs>)
TARGET_INCLUDE_DIRECTORIES(<target> <specifier> <dirs>)
\end{lstlisting}
		\item hodnota \texttt{specifier} může být
			\begin{itemize}
				\item \texttt{INTERFACE}
				\item \texttt{PUBLIC} - vystačíme si s touto
				\item \texttt{PRIVATE}
			\end{itemize}
		\item např. pro přidání složky \texttt{include}
\begin{lstlisting}[basicstyle=\fontsize{8}{9}\selectfont\ttfamily]
INCLUDE_DIRECTORIES(${INCLUDE_DIRECTORIES} include/)
TARGET_INCLUDE_DIRECTORIES(my-executable PUBLIC
     ${INCLUDE_DIRECTORIES} include/)
\end{lstlisting}
	\end{itemize}
\end{xframe}

\begin{xframe}{Linkování knihoven}
	\begin{itemize}
		\item linkování knihoven taktéž globálně / pro jeden cíl
		\item pozor - globálně je třeba před přidáním cílů
\begin{lstlisting}[basicstyle=\fontsize{8}{9}\selectfont\ttfamily]
LINK_LIBRARIES(<libraries>)
TARGET_LINK_LIBRARIES(<target> <libraries>)
\end{lstlisting}
		\item např.
\begin{lstlisting}[basicstyle=\fontsize{8}{9}\selectfont\ttfamily]
LINK_LIBRARIES(my-toolset)
TARGET_LINK_LIBRARIES(my-executable my-toolset)
\end{lstlisting}
	\end{itemize}
\end{xframe}

\begin{xframe}{Hledání externích knihoven}
	\begin{itemize}
		\item je potřeba modul CMake
		\item nějaké má CMake přibalené s sebou
		\item modul je možno vyvolat
\begin{lstlisting}[basicstyle=\fontsize{8}{9}\selectfont\ttfamily]
FIND_PACKAGE(<packagename> [flag])
\end{lstlisting}
		\item \texttt{flag} nás bude zajímat jen jedna: \texttt{REQUIRED} - nelze bez této knihovny pokračovat
		\item např.
\begin{lstlisting}[basicstyle=\fontsize{8}{9}\selectfont\ttfamily]
FIND_PACKAGE(OpenSSL REQUIRED)
\end{lstlisting}
		\item modul nastavuje proměnné
		\item ostatní silně specifické
	\end{itemize}
\end{xframe}

\begin{xframe}{Hledání externích knihoven}
	\begin{itemize}
		\item interní moduly CMake většinou nastavují nějak standardizované proměnné
		\item vždy \texttt{<packagename>\_FOUND} pokud se knihovnu podaří nalézt
		\item často \texttt{<packagename>\_INCLUDE\_DIR} - kde hledat hlavičkové soubory
		\item často \texttt{<packagename>\_LIBRARIES} - seznam knihoven k linkování
		\item všechny moduly, které s sebou CMake nese, jsou zdokumentovány: \url{https://cmake.org/cmake/help/latest/manual/cmake-modules.7.html}
	\end{itemize}
\end{xframe}

\begin{xframe}{Parametrizace překladu}
	\begin{itemize}
		\item často je potřeba předat nějaký přepínač pro parametrizaci sestavení
		\item opět lze globálně i pro target
\begin{lstlisting}[basicstyle=\fontsize{8}{9}\selectfont\ttfamily]
ADD_DEFINITIONS(<defines>)
TARGET_COMPILE_DEFINITIONS(<target> <specifier> <defines>)
\end{lstlisting}
		\item pozor, defines musí být včetně prefixu \texttt{-D} pro preprocesorové direktivy (aby se chovaly jako \texttt{\#define})
		\item např.
\begin{lstlisting}[basicstyle=\fontsize{8}{9}\selectfont\ttfamily]
ADD_DEFINITIONS(-DDEV_BUILD)
TARGET_COMPILE_DEFINITIONS(my-executable PUBLIC -DDEV_BUILD)
\end{lstlisting}
	\end{itemize}
\end{xframe}


\begin{xframe}{Vyhledání souborů}
	\begin{itemize}
		\item abychom nemuseli otrocky vypisovat seznam souborů, lze je nechat vyhledat
		\item po vyhledání jsou uloženy do proměnné
		\item využijeme příkaz \texttt{FILE}, který má širší využití
\begin{lstlisting}[basicstyle=\fontsize{8}{9}\selectfont\ttfamily]
FILE(<operation> <parameters>)
\end{lstlisting}
		\item pro vyhledání souborů v dané složce s příponou .cpp a .h např.:
\begin{lstlisting}[basicstyle=\fontsize{8}{9}\selectfont\ttfamily]
FILE(GLOB_RECURSE my_app_sources ./ *.cpp *.h)
\end{lstlisting}
		\item operace \texttt{GLOB\_RECURSE} projde i podsložky
		\item pokud nechceme, lze použít jen \texttt{GLOB}
	\end{itemize}
\end{xframe}

\begin{xframe}{Řídicí struktury}
	\begin{itemize}
		\item CMake podporuje všemožné řídicí struktury
		\item nejjednodušší je \texttt{IF}
		\item syntaxe se myšlenkou podobá podmínkám např. z bashe
\begin{lstlisting}[basicstyle=\fontsize{8}{9}\selectfont\ttfamily]
IF(<expression>)
    # something
ELSE()
    # something else
ENDIF()
\end{lstlisting}
		\item výrazem může být pouhá proměnná - pak se ověřuje její existence
		\item nebo např. porovnávací výraz
	\end{itemize}
\end{xframe}

\begin{xframe}{Řídicí struktury}
	\begin{itemize}
		\item příklad - podařilo se nalézt OpenSSL knihovnu?
\begin{lstlisting}[basicstyle=\fontsize{8}{9}\selectfont\ttfamily]
IF(OpenSSL_FOUND)
    # great!
ENDIF()
\end{lstlisting}
		\item příklad - je proměnná nastavena na hodnotu?
\begin{lstlisting}[basicstyle=\fontsize{8}{9}\selectfont\ttfamily]
IF(MY_VARIABLE STREQUAL "windows")
    # great!
ENDIF()
\end{lstlisting}
		\item a další, viz \url{https://cmake.org/cmake/help/latest/command/if.html}
	\end{itemize}
\end{xframe}

\begin{xframe}{Výstup do konzole}
	\begin{itemize}
		\item občas se hodí oznámit něco do konzole
\begin{lstlisting}[basicstyle=\fontsize{8}{9}\selectfont\ttfamily]
MESSAGE([mode] "message")
\end{lstlisting}
		\item módy
			\begin{itemize}
				\item \texttt{STATUS} - diagnostický výstup
				\item \texttt{WARNING}, \texttt{AUTHOR\_WARNING} - varování, nezastaví provádění
				\item \texttt{SEND\_ERROR} - chyba, nedovolí vygenerovat projekt
				\item \texttt{FATAL\_ERROR} - chyba, okamžitě zastaví provádění
				\item \texttt{DEPRECATION} - deprecated varování (zastaví, pokud je nastaveno)
			\end{itemize}
		\item např.
\begin{lstlisting}[basicstyle=\fontsize{8}{9}\selectfont\ttfamily]
MESSAGE(FATAL_ERROR "Could not find OpenSSL!")
\end{lstlisting}
	\end{itemize}
\end{xframe}

\begin{xframe}{Příklad}
	\begin{itemize}
		\item Prostor pro příklad 11\_cmake
	\end{itemize}
\end{xframe}

\section{Valgrind}
\subsection{Poznámky}

\begin{xframe}{Poznámky k memory leak Valgrind příkladům}
	\begin{itemize}
		\item byly připraveny příklady na chyby vedoucí k memory leakům, co mohou v C++ vzniknout
		\item oba zdůrazňují, proč je vhodné používat konstrukce moderního C++ (často založené na RAII)
		\item v příkladech je návod na sestavení i na spuštění memchecku
	\end{itemize}
\end{xframe}

\begin{xframe}{Příklad}
	\begin{itemize}
		\item Prostor pro příklad 12\_a\_leak a 12\_b\_leak
	\end{itemize}
\end{xframe}

\begin{xframe}{Poznámky k buffer overrun Valgrind příkladu}
	\begin{itemize}
		\item byl připraven příklad demonstrující chybu vedoucí k přesažení bufferu (pole, vyhrazené paměti, ..)
		\item opět zdůrazňuje důležitost např. STL containerů
		\item v příkladech je návod na sestavení i na spuštění sgchecku
	\end{itemize}
\end{xframe}

\begin{xframe}{Příklad}
	\begin{itemize}
		\item Prostor pro příklad 12\_c\_range
	\end{itemize}
\end{xframe}

\section{GDB}
\subsection{O GDB}

\begin{xframe}{GDB}
	\begin{itemize}
		\item zaměříme se na použití pro diagnostiku segfaultů
		\item jde samozřejmě o plnohodnotný debugger se spoustou funkcí
	\end{itemize}
\end{xframe}

\subsection{Použití}

\begin{xframe}{Použití}
	\begin{itemize}
		\item lze spustit program rovnou pod GDB
\begin{lstlisting}[basicstyle=\fontsize{8}{9}\selectfont\ttfamily]
gdb my-program
\end{lstlisting}
		\item to spustí prostředí GDB
		\item program lze spustit příkazem \texttt{run}
		\item zastavit lze stisknutím ctrl-C
		\item popř. breakpointem nebo signálem od systému (třeba segfault)
	\end{itemize}
\end{xframe}

\begin{xframe}{Core dump}
	\begin{itemize}
		\item nebo lze dovolit generování tzv. core dumpů
\begin{lstlisting}[basicstyle=\fontsize{8}{9}\selectfont\ttfamily]
ulimit -c <size>
\end{lstlisting}
		\item \texttt{size} je specifikátor velikosti nebo \texttt{unlimited}
		\item core dumpy jsou 1:1 dumpy paměti s diagnostickými informacemi, které umí GDB interpretovat
		\item po pádu programu se v konzoli objeví hláška \texttt{core dumped} (resp. \texttt{obraz paměti uložen}
		\item lze vyvolat
\begin{lstlisting}[basicstyle=\fontsize{8}{9}\selectfont\ttfamily]
gdb -c <coredump> <program>
\end{lstlisting}
		\item výchozí jméno je \texttt{core} a ukládá se k aplikaci
	\end{itemize}
\end{xframe}

\begin{xframe}{Příkazy}
	\begin{itemize}
		\item \texttt{run} - spuštění programu (od začátku)
		\item \texttt{break [where]} - nastavení breakpointu na místo (např. main.cpp:10)
		\item \texttt{continue} - pokračování po zastavení
		\item \texttt{backtrace [full]} - výstup zanoření v aktuálním vlákně (full = včetně všech kontextových informací)
		\item \texttt{frame [num]} - přesun do jiného frame zanoření v aktuálním kontextu
		\item \texttt{info threads} - informace o běžících vláknech
		\item \texttt{thread [num]} - přesun do kontextu jiného vlákna
		\item \texttt{print [symbol]} - vytištění symbolu (syntaxe jako v kódu)
		\item \texttt{quit} - ukončení GDB
		\item a samozřejmě spousty, spousty dalších, toto je jen základní sada
	\end{itemize}
\end{xframe}

\begin{xframe}{Příkazy}
	\begin{itemize}
		\item my budeme navíc potřebovat ještě
		\item \texttt{thread apply [thread list / all] command} - provedení příkazu nad více vlákny
		\item nejčastěji jako:
\begin{lstlisting}[basicstyle=\fontsize{8}{9}\selectfont\ttfamily]
thread apply all backtrace full
\end{lstlisting}
	\end{itemize}
\end{xframe}

\begin{xframe}{Příklad}
	\begin{itemize}
		\item Prostor pro příklady 13\_a\_segfault, 13\_b\_segfault a 13\_c\_deadlock
	\end{itemize}
\end{xframe}

\section{Perf}
\subsection{O perfu}

\begin{xframe}{Perf}
	\begin{itemize}
		\item diagnostický subsystém
		\item mj. použitelný i jako profiler
		\item integrovaný do jádra linuxu (perf events)
		\item neinvazivní metoda diagnostiky aplikace
		\item bohužel nutný superuser (\texttt{root})
		\item na Debian-based distribucích balíček \texttt{linux-tools}
	\end{itemize}
\end{xframe}

\subsection{Příkazy}

\begin{xframe}{Práce s perfem}
	\begin{itemize}
		\item lze buď aplikaci spustit pod perfem
\begin{lstlisting}[basicstyle=\fontsize{8}{9}\selectfont\ttfamily]
perf record ./aplikace
\end{lstlisting}
		\item nebo se lze připojit k již běžícímu procesu podle PIDu
\begin{lstlisting}[basicstyle=\fontsize{8}{9}\selectfont\ttfamily]
perf record -p 1234
\end{lstlisting}
		\item je vhodné připojit přepínač \texttt{-g} pro zaznamenání hierarchie volání
	\end{itemize}
\end{xframe}

\begin{xframe}{Eventy}
	\begin{itemize}
		\item perf zaznamenává vzorek (zanoření, aktuální adresa) v momentě, kdy přijde tzv. event
		\item event je propagován interní cestou přes jádro (perf events subsystém), HW eventy navíc pomocí přerušení
		\item defaultně se zaznamenávají cykly CPU (event \texttt{cycles})
		\item lze ale profilovat i počet cache-miss, syscallů, instrukcí, branch-miss a mnoho dalšího (celý seznam \texttt{perf list})
		\item přepínač \texttt{-e <event>}
	\end{itemize}
\end{xframe}

\begin{xframe}{Prohlížení}
	\begin{itemize}
		\item perf ukládá soubor \texttt{perf.data} do aktuálního adresáře
		\item v něm jsou zaznamenány i cesty k souborům s kódem, atd. - není proto potřeba být v žádném konkrétním adresáři
		\item data lze prohlédnout příkazem
\begin{lstlisting}[basicstyle=\fontsize{8}{9}\selectfont\ttfamily]
perf report
\end{lstlisting}
		\item popř. formou anotovaného kódu
\begin{lstlisting}[basicstyle=\fontsize{8}{9}\selectfont\ttfamily]
perf annotate
\end{lstlisting}
		\item nebo PIVO \url{https://github.com/ProjectPIVO} :)
	\end{itemize}
\end{xframe}

\begin{xframe}{Příklad}
	\begin{itemize}
		\item Prostor pro příklad 14\_a\_slow
	\end{itemize}
\end{xframe}

\section{Závěr}
\subsection{Závěr}

\begin{xframe}{Závěr}
	\begin{itemize}
		\item CMake, GDB, Valgrind a perf jsou obrovské nástroje
		\item probrali jsme přehled základních prvků
	\end{itemize}
\end{xframe}




\begin{xframe}{Konec 11., 12., 13. a 14. části}
\texttt{std::cout << "Děkuji za pozornost"~<< std::endl;}
\texttt{exit(0);}
\end{xframe}




\end{document}




