% !TEX TS-program = pdflatex
% !TEX encoding = UTF-8 Unicode

\documentclass{beamer}
\usepackage[czech]{babel}
\usepackage[utf8]{inputenc}
\usepackage{times}
\usepackage[T1]{fontenc}
\usepackage{verbatim}
\usepackage{listings}
\usepackage{xcolor}

\mode<presentation>
{
	\usetheme{antibes}
	\usecolortheme{orchid}
}

\setbeamertemplate{navigation symbols}{}

\definecolor{cmtgreen}{RGB}{0,192,0}

\begin{document}

\title{C++; delete Java;}
\subtitle{Část 10: speciality}
\author{Kennny}
\date{srpen 2017}

\frame{\titlepage}

\lstset{language=C++,
        basicstyle=\ttfamily,
        keywordstyle=\color{blue}\ttfamily,
        stringstyle=\color{red}\ttfamily,
        commentstyle=\color{cmtgreen}\ttfamily,
        morecomment=[l][\color{magenta}]{\#}
}

\newenvironment{xframe}[1][]
  {\begin{frame}[fragile,environment=xframe,#1]}
  {\end{frame}}

\begin{comment}
\begin{xframe}{tttt}
	\begin{itemize}
		\item
	\end{itemize}
\end{xframe}
\end{comment}



\section{Speciality}
\subsection{Jazyk}



\begin{xframe}{constexpr}
	\begin{itemize}
		\item klíčové slovo \texttt{constexpr}
		\item umožňuje vyhodnocení v čase kompilace, pokud to je možné
		\item pokud ne, vyhodnotí za běhu jako normálně
		\item \texttt{constexpr} proměnné a funkce
		\item hodí se např. na vyhodnocení magic constant nebo obecně sady hodnot, o které víme, že pro daný argument nikdy jiná nebude
	\end{itemize}
\end{xframe}

\begin{xframe}{constexpr}
	\begin{itemize}
		\item \texttt{constexpr} funkce funguje pouze, pokud je její návratová hodnota přiřazována do \texttt{constexpr} konstanty, jinak je vyhodnocena v čase běhu
	\end{itemize}
	
\begin{lstlisting}[basicstyle=\fontsize{8}{9}\selectfont\ttfamily]
constexpr int magicTransform(const int arg)
{
    return arg*5 + 10;
}

// provede se v case kompilace
constexpr int arg1 = magicTransform(15);
// provede se az za behu
int arg2 = magicTransform(99);
\end{lstlisting}
\end{xframe}

\begin{xframe}{Příklad}
	\begin{itemize}
		\item Prostor pro příklad 10\_a\_constexpr
	\end{itemize}
\end{xframe}


% TODO


\begin{xframe}{Příklad}
	\begin{itemize}
		\item Prostor pro příklad 10\_b\_lambda
	\end{itemize}
\end{xframe}

\begin{xframe}{Příklad}
	\begin{itemize}
		\item Prostor pro příklad 10\_c\_auto
	\end{itemize}
\end{xframe}

\begin{xframe}{Příklad}
	\begin{itemize}
		\item Prostor pro příklad 10\_d\_stl
	\end{itemize}
\end{xframe}

\begin{xframe}{Příklad}
	\begin{itemize}
		\item Prostor pro příklad 10\_e\_enum\_class
	\end{itemize}
\end{xframe}

\begin{xframe}{Příklad}
	\begin{itemize}
		\item Prostor pro příklad 10\_f\_range\_based\_for
	\end{itemize}
\end{xframe}




\begin{xframe}{Konec 10. části}
\texttt{exit(0);}
\end{xframe}




\end{document}




