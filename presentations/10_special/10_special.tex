% !TEX TS-program = pdflatex
% !TEX encoding = UTF-8 Unicode

\documentclass{beamer}
\usepackage[czech]{babel}
\usepackage[utf8]{inputenc}
\usepackage{times}
\usepackage[T1]{fontenc}
\usepackage{verbatim}
\usepackage{listings}
\usepackage{xcolor}

\mode<presentation>
{
	\usetheme{antibes}
	\usecolortheme{orchid}
}

\setbeamertemplate{navigation symbols}{}

\definecolor{cmtgreen}{RGB}{0,192,0}

\begin{document}

\title{C++; delete Java;}
\subtitle{Část 10: speciality}
\author{Kennny}
\date{srpen 2017}

\frame{\titlepage}

\lstset{language=C++,
        basicstyle=\ttfamily,
        keywordstyle=\color{blue}\ttfamily,
        stringstyle=\color{red}\ttfamily,
        commentstyle=\color{cmtgreen}\ttfamily,
        morecomment=[l][\color{magenta}]{\#}
}

\newenvironment{xframe}[1][]
  {\begin{frame}[fragile,environment=xframe,#1]}
  {\end{frame}}

\begin{comment}
\begin{xframe}{tttt}
	\begin{itemize}
		\item
	\end{itemize}
\end{xframe}
\end{comment}



\section{Speciality}
\subsection{Jazyk}



\begin{xframe}{constexpr}
	\begin{itemize}
		\item klíčové slovo \texttt{constexpr}
		\item umožňuje vyhodnocení v čase kompilace, pokud to je možné
		\item pokud ne, vyhodnotí za běhu jako normálně
		\item \texttt{constexpr} proměnné a funkce
		\item hodí se např. na vyhodnocení magic constant nebo obecně sady hodnot, o které víme, že pro daný argument nikdy jiná nebude
	\end{itemize}
\end{xframe}

\begin{xframe}{constexpr}
	\begin{itemize}
		\item \texttt{constexpr} funkce funguje pouze, pokud je její návratová hodnota přiřazována do \texttt{constexpr} konstanty, jinak je vyhodnocena v čase běhu
	\end{itemize}
	
\begin{lstlisting}[basicstyle=\fontsize{8}{9}\selectfont\ttfamily]
constexpr int magicTransform(const int arg)
{
    return arg*5 + 10;
}

// provede se v case kompilace
constexpr int arg1 = magicTransform(15);
// provede se az za behu
int arg2 = magicTransform(99);
\end{lstlisting}
\end{xframe}

\begin{xframe}{Příklad}
	\begin{itemize}
		\item Prostor pro příklad 10\_a\_constexpr
	\end{itemize}
\end{xframe}


\begin{xframe}{Lambda funkce}
	\begin{itemize}
		\item \texttt{\#include <functional>}
		\item anonymní funkce
		\item šablonový typ \texttt{std::function}
		\item lze definovat vztah s vnější scope pomocí \emph{capture} bloku
			\begin{itemize}
				\item předávání hodnotou (kopií) (\texttt{=}) nebo referencí (\texttt{\&})
				\item předávání selektivně nebo všeho
			\end{itemize}
		\item parametry funkce standardně
		\item návratová hodnota určena typem v šabloně, dedukována, popř. určena
	\end{itemize}
\end{xframe}

\begin{xframe}{Příklady}
	\begin{itemize}
		\item bezparametrická, bez capture, bez návratové hodnoty
\begin{lstlisting}[basicstyle=\fontsize{8}{9}\selectfont\ttfamily]
std::function<void()> fn = []() { };
\end{lstlisting}
		\item bezparametrická, capture všeho hodnotou, vrací integer
\begin{lstlisting}[basicstyle=\fontsize{8}{9}\selectfont\ttfamily]
std::function<int()> fn = [=]() -> int { return 5; };
auto fn = [=]() { return (int)5; };
\end{lstlisting}
		\item parametry, capture specifický, návratová hodnota dedukovaná
\begin{lstlisting}[basicstyle=\fontsize{8}{9}\selectfont\ttfamily]
auto fn = [a, b, &c, this](int p1, float p2) {
    return p1*a + b + c.x*p2 + getSomething();
};
\end{lstlisting}
	\end{itemize}
\end{xframe}

\begin{xframe}{Příklady}
	\begin{itemize}
		\item často použití v konkrétním kontextu
		\item vláknová funkce
\begin{lstlisting}[basicstyle=\fontsize{8}{9}\selectfont\ttfamily]
std::thread thr1([&]() {
    while (running)
        doSomething();
});
\end{lstlisting}
		\item predikát pro řazení
\begin{lstlisting}[basicstyle=\fontsize{8}{9}\selectfont\ttfamily]
std::sort(a.begin(), a.end(), [](int x, int y) {
    return x < y;
});
\end{lstlisting}
		\item a další...
	\end{itemize}
\end{xframe}

\begin{xframe}{Vsuvka: std::bind}
	\begin{itemize}
		\item v STL modulu \emph{functional} dále možnost svázat volání funkce s parametrem
		\item hodí se např. pokud víme, že budeme určitou funkci volat stále se stejným parametrem
		\item pro ostatní parametry musíme při vytváření bindingu použít tzv. placeholder
		\item výsledný typ pak přetěžuje operátor() a odstiňuje funkční volání
	\end{itemize}
\end{xframe}

\begin{xframe}{Vsuvka: std::bind}
	\begin{itemize}
		\item např. bind parametru funkce powf pro druhou mocninu - exponent vždy 2
		\item funkce \texttt{powf} má dva argumenty: základ (ten chceme proměnný) a exponent (ten chceme vždy 2)
		\item proto za základ vložíme placeholder a za exponent konstantu
\begin{lstlisting}[basicstyle=\fontsize{8}{9}\selectfont\ttfamily]
auto druha_mocnina =
     std::bind(powf, std::placeholders::_1, 2.0f);
\end{lstlisting}
		\item nyní lze volat
\begin{lstlisting}[basicstyle=\fontsize{8}{9}\selectfont\ttfamily]
// ekvivalent k powf(15.0f, 2.0f);
float kv = druha_mocnina(15.0f);
\end{lstlisting}		
	\end{itemize}
\end{xframe}

\begin{xframe}{Vsuvka: std::bind}
	\begin{itemize}
		\item nelze spolehlivě říct, kdy je lepší bind, a kdy funkce / lambda
		\item většinou jde o konkrétní případy
		\item hraje roli pouze čistota kódu a styl, než výkon nebo možnosti
	\end{itemize}
\end{xframe}

\begin{xframe}{Příklad}
	\begin{itemize}
		\item Prostor pro příklad 10\_b\_lambda
	\end{itemize}
\end{xframe}

\begin{xframe}{Automatická dedukce typu}
	\begin{itemize}
		\item klíčové slovo \texttt{auto}
		\item lze dedukovat typ proměnné, návratový typ funkce, šablonový typ, a další...
		\item nám bude stačit typ proměnné a návratový typ funkce
		\item někdy se velmi hodí - šetří psaní a pamatování
		\item např. u \texttt{std::bind} je to více než vhodné
	\end{itemize}
\end{xframe}

\begin{xframe}{Automatická dedukce typu}
	\begin{itemize}
		\item dedukce typu proměnné
			\begin{itemize}
				\item podle přiřazované hodnoty
				\item podle návratové hodnoty funkce
			\end{itemize}
		\item dedukce návratové hodnoty
			\begin{itemize}
				\item podle \texttt{return} statementu
				\item explicitním určením syntaxí s \texttt{->} (např. částečné dourčení typu, vhodné u lambda funkcí)
			\end{itemize}
	\end{itemize}
\end{xframe}

\begin{xframe}{Automatická dedukce typu}
	\begin{itemize}
		\item dedukce u proměnných je většinou úspěšná
\begin{lstlisting}[basicstyle=\fontsize{8}{9}\selectfont\ttfamily]
auto val = 5; // dedukovan integer
auto valf = 1.2f // dedukovan float
auto str = "abcd"; // dedukovan const char*
auto kv = powf(5.0f, 2.0f); // dedukovan float
\end{lstlisting}
		\item u funkcí existují případy jednoduché (jeden return, není jiná možnost jak dedukovat)
\begin{lstlisting}[basicstyle=\fontsize{8}{9}\selectfont\ttfamily]
auto getPositionX()
{
    return mPositionX;
}
\end{lstlisting}
	\end{itemize}
\end{xframe}

\begin{xframe}{Automatická dedukce typu}
	\begin{itemize}
		\item ...i případy složitější
\begin{lstlisting}[basicstyle=\fontsize{8}{9}\selectfont\ttfamily]
auto getSomething()
{
    if (condition)
        return nullptr;
    if (otherCondition)
        return mMember;
    return std::get<0>(sometuple);
}
\end{lstlisting}
		\item syntaxi lze doplnit o dedukci typu
\begin{lstlisting}[basicstyle=\fontsize{8}{9}\selectfont\ttfamily]
auto getSomething() -> MyType*
\end{lstlisting}
		\item má smysl např. u lambda funkcí - šetří psaní std::function<..>
\begin{lstlisting}[basicstyle=\fontsize{8}{9}\selectfont\ttfamily]
auto myLambda = []() -> int { return a ? 2 : 9.0; };
\end{lstlisting}
	\end{itemize}
\end{xframe}

\begin{xframe}{Příklad}
	\begin{itemize}
		\item Prostor pro příklad 10\_c\_auto
	\end{itemize}
\end{xframe}

\subsection{STL knihovna}

\begin{xframe}{Algoritmy STL knihovny}
	\begin{itemize}
		\item \texttt{\#include <algorithm>}
		\item obsahuje základní sadu algoritmů, která se může kdykoliv hodit
		\item jejich implementace je závislá na dodavateli standardní knihovny (ostatně jako vše ve standardní knihovně)
		\item můžeme se ale pravděpodobně spolehnout na kvalitní implementaci, co nejblíže optimu
	\end{itemize}
\end{xframe}

\begin{xframe}{Algoritmy STL knihovny}
	\begin{itemize}
		\item algoritmů je přítomno spousty, budou uvedeny jen někteří zástupci
		\item práce s algoritmy pak odráží jednotné schéma
		\item často je vstupem rozsah iterátorů, parametr a např. funkce/funktor/lambda
	\end{itemize}
\end{xframe}

\begin{xframe}{Algoritmy STL knihovny}
	\begin{itemize}
		\item \texttt{std::generate} - do daného rozsahu nageneruje danou funkcí hodnoty (náhrada za for cyklus)
		\item \texttt{std::shuffle} - náhodné rozmíchání daného rozsahu
		\item \texttt{std::move} - provede přesun obsahu objektu (rozsahu) do druhého (např. přesun vlastnictví unique\_ptr
		\item \texttt{std::copy} - kopie objektu (rozsahu) do druhého
		\item \texttt{std::for\_each} - nad zadaným rozsahem provede danou funkci
		\item \texttt{std::sort} - seřadí rozsah, můžeme dodat i funkci pro řazení
	\end{itemize}
\end{xframe}

\begin{xframe}{Algoritmy STL knihovny}
	\begin{itemize}
		\item \texttt{std::count\_if} - spočítá prvky splňující kritérium (funkce)
		\item \texttt{std::min\_element}, \texttt{std::max\_element} - najde minimální/maximální prvek
		\item \texttt{std::minmax\_element} - sdružené předchozí funkce
		\item \texttt{std::prev\_permutation}, \texttt{std::next\_permutation} - generování permutací
		\item a mnoho dalších... \url{http://en.cppreference.com/w/cpp/header/algorithm}
	\end{itemize}
\end{xframe}

\begin{xframe}{Příklad}
	\begin{itemize}
		\item Prostor pro příklad 10\_d\_stl
	\end{itemize}
\end{xframe}

\subsection{Jazyk}

\begin{xframe}{enum class}
	\begin{itemize}
		\item klíčové slovo \texttt{class} u výčtového typu
		\item výčtový typ se tímto stane tzv. silně typovaný
		\item chová se proto jako samostatný typ, ne alias pro integer
		\item hodnotu je nutné vždy uvodit názvem výčtového typu
\begin{lstlisting}[basicstyle=\fontsize{8}{9}\selectfont\ttfamily]
enum class MyEnum
{
    Value1,
    Value2,
    Value3
};

MyEnum val = MyEnum::Value1;
\end{lstlisting}
	\end{itemize}
\end{xframe}

\begin{xframe}{enum class}
	\begin{itemize}
		\item lze však explicitně převést typ na integer a nazpátek, jde jen o syntaktickou kontrolu
\begin{lstlisting}[basicstyle=\fontsize{8}{9}\selectfont\ttfamily]
enum class MyEnum
{
    Value1,
    Value2,
    Value3
};

int val = (int)MyEnum::Value1;
MyEnum val2 = (MyEnum)val;
\end{lstlisting}
	\end{itemize}
\end{xframe}

\begin{xframe}{Příklad}
	\begin{itemize}
		\item Prostor pro příklad 10\_e\_enum\_class
	\end{itemize}
\end{xframe}

\begin{xframe}{Range-based for}
	\begin{itemize}
		\item "chytrý"~for cyklus, který si je vědom rozsahů containerů
		\item nebo lépe.. je si vědom, že se často iteruje "od \texttt{begin()} až do \texttt{end()}"
		\item dovoluje tedy projít celý container užitím zkrácené syntaxe, např.
\begin{lstlisting}[basicstyle=\fontsize{8}{9}\selectfont\ttfamily]
for (int &hodnota : intVektor)
{
    // do something
}
\end{lstlisting}
	\end{itemize}
\end{xframe}

\begin{xframe}{Range-based for}
	\begin{itemize}
		\item zajímat nás budou obměny deklarace prvku (levá část)
		\item můžeme přejímat hodnoty hodnotou (kopií) a referencí
		\item některé containery vyžadují, aby byly hodnoty \texttt{const} (např. mapa nebo množina)
		\item lze použít automatickou dedukci typu, zda to bude reference/hodnota ale musíme rozhodnout my
\begin{lstlisting}[basicstyle=\fontsize{8}{9}\selectfont\ttfamily]
for (auto hodnota : container)
{
    // do something
}

for (auto &hodnota : container)
{
    // do something
}
\end{lstlisting}
	\end{itemize}
\end{xframe}

\begin{xframe}{Range-based for}
	\begin{itemize}
		\item \texttt{std::map} takto vrací instanci \texttt{std::pair}
		\item první prvek páru musí být vždy \texttt{const}
			\begin{itemize}
				\item ze zjevných důvodů - přímá modifikace klíče by mohla poškodit integritu kontejneru
			\end{itemize}
\begin{lstlisting}[basicstyle=\fontsize{8}{9}\selectfont\ttfamily]
for (std::pair<const int, int> par : mapa)
{
    // do something
}
\end{lstlisting}
		\item je možná i const reference - pak ale nelze měnit ani hodnotu
\begin{lstlisting}[basicstyle=\fontsize{8}{9}\selectfont\ttfamily]
for (std::pair<int, int> const& par : mapa)
{
    // do something
}
\end{lstlisting}
	\end{itemize}
\end{xframe}

\begin{xframe}{Range-based for}
	\begin{itemize}
		\item \texttt{std::set} sice vrací prvek jako takový, ale vždy kopií nebo \texttt{const} referencí
		\item opět ze zjevných důvodů - prvek je klíčován sám sebou
\begin{lstlisting}[basicstyle=\fontsize{8}{9}\selectfont\ttfamily]
for (int par : mapa)
{
    // do something
}

for (int const& par : mapa)
{
    // do something
}
\end{lstlisting}
	\end{itemize}
\end{xframe}

\begin{xframe}{Příklad}
	\begin{itemize}
		\item Prostor pro příklad 10\_f\_range\_based\_for
	\end{itemize}
\end{xframe}

\section{Kam dál?}

\subsection{Doporučení}

\begin{xframe}{Konec}
	\begin{itemize}
		\item C++ jazyk a STL jsou obří, zdaleka jsme neobsáhli všechno
		\item další studium, zajímavé věci...
	\end{itemize}
\end{xframe}

\begin{xframe}{Další studium}
	\begin{itemize}
		\item regulární výrazy
		\item synchronizace vláken - čtenář-písař, různé atomic containery
		\item další možnosti random enginů
		\item zbytek \texttt{<algorithm>} hlavičky
		\item šablonové porno (static\_assert, variable and variadic templates, ..)
		\item RTTI, decltype a typeid
		\item UTF-8
		\item user-defined literals
		\item zbytek \texttt{std::chrono}
	\end{itemize}
\end{xframe}

\begin{xframe}{Další studium}
	\begin{itemize}
		\item atributy ( [[deprecated]], ..)
		\item multimap, multiset
		\item C++17
		\item ...
	\end{itemize}
\end{xframe}





\begin{xframe}{Konec 10. části}
\texttt{std::cout << "Děkuji za pozornost"~<< std::endl;}
\texttt{exit(0);}
\end{xframe}




\end{document}




