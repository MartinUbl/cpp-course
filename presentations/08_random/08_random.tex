% !TEX TS-program = pdflatex
% !TEX encoding = UTF-8 Unicode

\documentclass{beamer}
\usepackage[czech]{babel}
\usepackage[utf8]{inputenc}
\usepackage{times}
\usepackage[T1]{fontenc}
\usepackage{verbatim}
\usepackage{listings}
\usepackage{xcolor}

\mode<presentation>
{
	\usetheme{antibes}
	\usecolortheme{orchid}
}

\setbeamertemplate{navigation symbols}{}

\definecolor{cmtgreen}{RGB}{0,192,0}

\begin{document}

\title{C++; delete Java;}
\subtitle{Část 8: pseudonáhodná čísla}
\author{Kennny}
\date{srpen 2017}

\frame{\titlepage}

\lstset{language=C++,
        basicstyle=\ttfamily,
        keywordstyle=\color{blue}\ttfamily,
        stringstyle=\color{red}\ttfamily,
        commentstyle=\color{cmtgreen}\ttfamily,
        morecomment=[l][\color{magenta}]{\#}
}

\newenvironment{xframe}[1][]
  {\begin{frame}[fragile,environment=xframe,#1]}
  {\end{frame}}

\begin{comment}
\begin{xframe}{tttt}
	\begin{itemize}
		\item
	\end{itemize}
\end{xframe}
\end{comment}



\section{Generátory pseudonáhodných čísel}
\subsection{Obecné}



\begin{xframe}{Generátory}
	\begin{itemize}
		\item je k dispozici několik generátorů
		\item tyto generátory se vyznačují tím, že každý generuje "jinou náhodu"
		\item pro nás je důležité jen to, že existují
		\item \texttt{std::random\_device} - obaluje zdroj náhody, např. \texttt{/dev/urandom} pokud je k dispozici, apod.
		\item může být ale pomalý, hodí se např. jen pro inicializaci generátoru pseudonáhodných čísel
	\end{itemize}
\end{xframe}

\begin{xframe}{Generátory}
	\begin{itemize}
		\item v STL existuje více generátorů pseudonáh. čísel
			\begin{itemize}
				\item \texttt{std::default\_random\_engine} - alias pro jiný generátor
				\item \texttt{std::mersenne\_twister\_engine} - Mersenne Twister generátor
				\item \texttt{std::linear\_congruental\_engine} - lineární kongruentní generátor
				
			\end{itemize}
		\item ty mají často ale své parametry
		\item na to v STL bylo myšleno a byly vytvořeny aliasy s ověřenou parametrizací
			\begin{itemize}
				\item \texttt{std::mt19937} - většinou výchozí, mersenne twister
				\item \texttt{std::minstd\_rand}
				\item \texttt{std::knuth\_b}
				\item ...
			\end{itemize}
	\end{itemize}
\end{xframe}


\begin{xframe}{Rozložení}
	\begin{itemize}
		\item samotný generátor nestačí
		\item jím generované hodnoty musí projít ještě generátorem rozložení
		\item k dispozici jsou standardní rozložení
			\begin{itemize}
				\item \texttt{std::uniform\_int\_distribution} - rovnoměrné celočíselné
				\item \texttt{std::uniform\_real\_distribution} - rovnoměrné desetinné
				\item \texttt{std::normal\_distribution} - normální
				\item \texttt{std::poisson\_distribution} - Poissonovo
				\item \texttt{std::chi\_squared\_distribution} - Chí-kvadrát
				\item a spousty dalších
			\end{itemize}
		\item šablonové typy
	\end{itemize}
\end{xframe}

\begin{xframe}{Příklad}
	\begin{itemize}
		\item prakticky vždy si vystačíme s konstrukcí podobnou této
	\end{itemize}
	
\begin{lstlisting}[basicstyle=\fontsize{8}{9}\selectfont\ttfamily]
// klidne jen jeden na cely program
std::random_device rd;

// vytvoreni PRNG generatoru, jako seed je pouzita hodnota
// z random device
std::default_random_engine eng(rd());
std::uniform_int_distribution<int> dist(1, 6);

int hraciKostka()
{
    return dist(eng);
}
\end{lstlisting}
	
\end{xframe}


\begin{xframe}{Příklad}
	\begin{itemize}
		\item Prostor pro příklad 08\_a\_random
	\end{itemize}
\end{xframe}




\begin{xframe}{Konec 8. části}
\texttt{exit(0);}
\end{xframe}




\end{document}




