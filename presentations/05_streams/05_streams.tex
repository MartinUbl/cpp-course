% !TEX TS-program = pdflatex
% !TEX encoding = UTF-8 Unicode

\documentclass{beamer}
\usepackage[czech]{babel}
\usepackage[utf8]{inputenc}
\usepackage{times}
\usepackage[T1]{fontenc}
\usepackage{verbatim}
\usepackage{listings}
\usepackage{xcolor}

\mode<presentation>
{
	\usetheme{antibes}
	\usecolortheme{orchid}
}

\setbeamertemplate{navigation symbols}{}

\definecolor{cmtgreen}{RGB}{0,192,0}

\begin{document}

\title{C++; delete Java;}
\subtitle{Část 5: I/O operace a streamy}
\author{Kennny}
\date{srpen 2017}

\frame{\titlepage}

\lstset{language=C++,
        basicstyle=\ttfamily,
        keywordstyle=\color{blue}\ttfamily,
        stringstyle=\color{red}\ttfamily,
        commentstyle=\color{cmtgreen}\ttfamily,
        morecomment=[l][\color{magenta}]{\#}
}

\newenvironment{xframe}[1][]
  {\begin{frame}[fragile,environment=xframe,#1]}
  {\end{frame}}

\begin{comment}
\begin{xframe}{tttt}
	\begin{itemize}
		\item
	\end{itemize}
\end{xframe}
\end{comment}



\section{I/O}
\subsection{Obecné}


\begin{xframe}{I/O operace}
	\begin{itemize}
		\item STL zaobaluje vstupně-výstupní operace do tzv. streamů
		\item dva streamy již známe - \texttt{std::cout} a \texttt{std::cin}
		\item mají přetížený operátor \texttt{<<} (výstup) a/nebo \texttt{>>} (vstup)
	\end{itemize}
\end{xframe}

\begin{xframe}{Stream}
	\begin{itemize}
		\item kromě jiných části hierarchie nás zajímají dva předci streamů
			\begin{itemize}
				\item \texttt{std::istream} - předek vstupního streamu
				\item \texttt{std::ostream} - předek výstupního streamu
			\end{itemize}
		\item již lze odvodit, že
			\begin{itemize}
				\item \texttt{std::cout} je globální instance \texttt{std::ostream}
				\item \texttt{std::cin} je globální instance \texttt{std::istream}
			\end{itemize}
		\item od těchto předků pak dědí další varianty, nás budou zajímat zejména varianty pro práci s řetězcem a souborem
	\end{itemize}
\end{xframe}

\begin{xframe}{Vlastní přetížení operátoru}
	\begin{itemize}
		\item je vhodné připomenout, že si můžeme přetížit operátor<< a >> jak potřebujeme pro práci s \texttt{std::istream} a \texttt{std::ostream}
		\item připomeňme například ten pro výpis souřadnic vektoru
\begin{lstlisting}[basicstyle=\fontsize{8}{9}\selectfont\ttfamily]
std::ostream& operator<<(std::ostream& os, Vektor const& v)
{
    os << "(" << v.x << ", " << v.y << ")";
    return os;
}
\end{lstlisting}
		\item nutno dodat, že operaci definujeme nad \texttt{std::ostream}, tedy bude fungovat jak nad řetězcovým, tak souborovým, tak jakýmkoliv jiným výstupním streamem
	\end{itemize}
\end{xframe}


\begin{xframe}{Input stream}
	\begin{itemize}
		\item při čtení je výchozí oddělovač zakončení řádku
		\item lze vynutit jiné použitím \texttt{std::getline} namísto operátoru \texttt{>>}
\begin{lstlisting}[basicstyle=\fontsize{8}{9}\selectfont\ttfamily]
std::getline(stream, targetstring, ';');
\end{lstlisting}
		\item má přetíženy operátory pro všechny primitivní typy a string
	\end{itemize}
\end{xframe}

\subsection{Konkrétní implementace}

\begin{xframe}{Řetězcový stream}
	\begin{itemize}
		\item \texttt{\#include <sstream>}
		\item \texttt{std::istringstream} a \texttt{std::ostringstream} - implementace \texttt{std::istream} a \texttt{std::ostream} nad řetězcem
		\item \texttt{std::stringstream} - vícenásobná dědičnost od obou předků
		\item pro vyzvednutí řetězce z \texttt{std::ostringstream}
			\begin{itemize}
				\item metoda \texttt{str()}, vrací instanci \texttt{std::string}
				\item pozor, nutno zkopírovat, po zaniknutí streamu zaniká i instance stringu
			\end{itemize}
	\end{itemize}
\end{xframe}

\begin{xframe}{Příklad}
	\begin{itemize}
		\item Prostor pro příklad 05\_a\_stringstream
	\end{itemize}
\end{xframe}


\begin{xframe}{Souborový stream}
	\begin{itemize}
		\item \texttt{\#include <fstream>}
		\item \texttt{std::ifstream} a \texttt{std::ofstream} - implementace \texttt{std::istream} a \texttt{std::ostream} nad souborem
		\item \texttt{std::fstream} - vícenásobná dědičnost od obou předků
		\item módy otevření souboru (bitmaska, lze skládat \texttt{|} )
			\begin{itemize}
				\item \texttt{std::ios::in} - pro čtení ("r")
				\item \texttt{std::ios::out} - pro zápis ("w")
				\item \texttt{std::ios::binary} - binárně ("b")
				\item \texttt{std::ios::app} - nepřepisovat, připojit za konec ("a")
			\end{itemize}
		\item RAII struktury - destruktor zavírá soubor
	\end{itemize}
\end{xframe}

\begin{xframe}{Souborový stream}
	\begin{itemize}
		\item pro textové čtení a zápis operátory \texttt{<<} a \texttt{>>}
		\item pro binární čtení a zápis metody \texttt{read()} a \texttt{write()}
	\end{itemize}
\end{xframe}

\begin{xframe}{Příklad}
	\begin{itemize}
		\item Prostor pro příklad 05\_b\_files
	\end{itemize}
\end{xframe}

\subsection{Manipulátory}

\begin{xframe}{Manipulátory výstupů}
	\begin{itemize}
		\item \texttt{\#include <iomanip>}
		\item vkládají se operátory \texttt{<<} nebo \texttt{>>} (podle typu streamu)
		\item parametrické a bezparametrické manipulátory
	\end{itemize}
\end{xframe}

\begin{xframe}{Číselný vstup/výstup}
	\begin{itemize}
		\item prefixovat číslo základem? \texttt{std::showbase} ; rozmysleli jsme si to? \texttt{std::noshowbase}
		\item změna základu čísla: \texttt{std::setbase()} (podporuje 8, 10 a 16), případně:
			\begin{itemize}
				\item \texttt{std::hex}, \texttt{std::dec}, \texttt{std::oct}
			\end{itemize}
\begin{lstlisting}[basicstyle=\fontsize{8}{9}\selectfont\ttfamily]
std::cout << std::hex  << std::showbase<< 127;
// vystup: 0x7F
\end{lstlisting}

		\item formát čísla s des. tečkou: \texttt{std::fixed}, \texttt{std::scientific}
		\item přesnost čísla s des. tečkou (počet des. míst): \texttt{std::setprecision}
\begin{lstlisting}[basicstyle=\fontsize{8}{9}\selectfont\ttfamily]
std::cout << std::scientific << std::setprecision(2) << 12.3531;
// vystup: 1.23e+01
\end{lstlisting}
	\end{itemize}
\end{xframe}

\begin{xframe}{Další}
	\begin{itemize}
		\item znaky čísla mají být velké? \texttt{std::uppercase} ; rozmysleli jsme si to? \texttt{std::nouppercase}
		\item chceme všechna čísla stejně dlouhá? \texttt{std::setw}
		\item chceme neobsazené cifry vypnit znakem? \texttt{std::setfill}
		\item bool hodnoty řetězcem? \texttt{std::boolalpha} ; rozmysleli jsme si to? \texttt{std::noboolalpha}
		\item a další... \url{http://en.cppreference.com/w/cpp/io/manip}
	\end{itemize}
\end{xframe}

\begin{xframe}{Příklad}
	\begin{itemize}
		\item Prostor pro příklad 05\_c\_iomanip
	\end{itemize}
\end{xframe}



\begin{xframe}{Konec 5. části}
\texttt{exit(0);}
\end{xframe}




\end{document}




